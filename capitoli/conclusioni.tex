Nel presente elaborato si è descritta l'evoluzione della traiettoria di 
particelle di materia prima del primo shell-crossing, ovvero nell'approssimazione di
Zel'dovich, ma anche negli istanti successivi (evoluzione post-Zel'dovich), limitatamente al caso unidimensionale. 
In particolare si è giunti a scrivere un'espressione per il dislocamento $\xi(q, \tau) = x(q,\tau)-q$
(e quindi anche per la mappa $q\mapsto x$) sia nel regime di single-stream che in quello di multistreaming.
In particolare, la definizione di $\xi_{PZA}$ in è stata fornita suddivisa in diverse regioni, permettendo così 
una ben precisa categorizzazione delle singolarità, come esposto in sezione \ref{sec:sing}: si è 
parlato di due tipi diversi di singolarità che possiamo definire locali, uno interno alle 
regioni di definizione, uno legato all'imposizione della continuità tra due regioni consecutive.
Vi è tuttavia un terzo tipo di singolarità, globale, e riguarda la scelta della
costante di integrazione $\xi_c(\tau)$ in (\ref{eqn:Fequation}).

\begin{comment}
    qui non ben chiaro cosa si intende per spatial average, e che tipo di singolarità è la terza
\end{comment}

La determinazione di $\xi_c(\tau)$ è di fondamentale importanza in quanto tale costante 
di integrazione rappresenta il boost galileiano che agisce sulle posizioni di tutte le 
particelle nella fase \textit{descending} del regime di multistreaming.

E' importante ribadire che la trattazione esposta in questo documento
è tutta nel caso unidimensionale (\cite{rampf}): ad ogni modo, le equazioni di base (\ref{eqn:cont_new}),
(\ref{eqn:poisson_new}), (\ref{eqn:density}) e (\ref{eqn:com}) sono facilmente estendibili
anche al caso tridimensionale. Sarà inoltre necessario utilizzare LPT a ordini superiori,
con l'accortezza di scegliere opportune condizioni al contorno per lo shell-crossing.
Una teoria pienamente sviluppata, basata quindi sull'equazione di Vlasov Poisson 
(\ref{eqn:vlasov}), darebbe accesso ad avanzate predizioni teoriche. In particolare è possibile
fornire i cosiddetti \textit{theoretical inputs} per la scrittura di teorie efficaci di uso 
frequente, spalleggiate da simulazioni a N corpi.

L'equazione di Vlasov-Poisson offre varie applicazioni aldilà dell'utilizzo in 
cosmologia: difatti ha un ruolo chiave in fisica dei plasmi.