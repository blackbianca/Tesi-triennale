La definizione di $\xi_{PZA}$ in regioni di validità ha permesso di guidare una ben precisa
categorizzazione delle singolarità, come esposto nella sezione \ref{sec:sing}: si è 
parlato di due tipi diversi di singolarità che possiamo definite locali, uno interno alle 
regioni di definizione, uno derivato dal tentativo di imporre le condizioni di continuità 
tra due regioni consecutive.
Vi è tuttavia un terzo tipo di singolarità, di tipo globale, ed è legato alla scelta della
costante di integrazione $\xi_c(\tau)$ in (\ref{eqn:Fequation}).

\begin{comment}
    qui non ben chiaro cosa si intende per spatial average
\end{comment}

La determinazione di $\xi_c(\tau)$ è di fondamentale importanza in quanto tale costante 
di integrazione rappresenta il boost galileiano che agisce sulle posizioni di tutte le 
particelle nella fase \textit{descending} del regime di multistreaming.

E' importante ribadire che la trattazione esposta in questo elaborato e in \cite{rampf}
è tutta nel caso unidimensionale: ad ogni modo, le equazioni di base (\ref{eqn:cont_new}),
(\ref{eqn:poisson_new}), (\ref{eqn:density}) e (\ref{eqn:com}) sono facilmente estendibili
anche al caso tridimensionale. Sarà inoltre necessario utilizzare LPT a ordini superiori,
con l'accortezza di scegliere opportune condizioni al contorno per lo shell-crossing.
Una teoria pienamente sviluppata, basata quindi sull'equazione di Vlasov Poisson 
(\ref{eqn:vlasov}), darebbe accesso ad avanzate predizioni teoriche. In particolre è possibile
fornire i cosiddetti \textit{theoretical inputs} per la scrittura di teorie efficaci di uso 
frequente, spalleggiate da simulazioni a N corpi.

L'equazione di Vlasov-Poisson offre varie applicazioni aldilà dell'utilizzo in 
cosmologia: difatti ha un ruolo chiave in fisica dei plasmi.