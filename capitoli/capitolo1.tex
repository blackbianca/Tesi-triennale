\section{Parte prima}

Le singolarità sono una caratteristica essenziale di molte discipline, e hanno talvolta conseguenze
drammatiche sul modello matematico\dots

In cosmologia, le caustiche sono i processi fondanti della formazione delle strutture cosmologiche
su larga scala. Queste corrispondono ai punti presso i quali la densità risulta localmente infinita
e sono i prodotti del cosiddetto \textit{shell-crossing}, che si verifica all'incrociarsi delle traiettorie
delle particelle di CDM, e corrispondono dunque alle singolarità matematiche che emergono naturalmente
nel trattare il problema con la dinamica gravitazione Newtoniana.
In particolare con al primo shell-crossing le particelle si spostano dal regime di \textit{single-stream}
a quello di \textit{multistreaming}, dove le orbite particellari si intersecano e i fenomeni di shell-crossing
si moltiplicano, determinando il diramarsi dei flussi.

Nel regime di single-stream l'approssimazione di Zel'dovich è opportuna, in quanto costitusice un modello non
accelerato dove le particelle procedono imperturbate e a velocità costante (quale velocità, comovente?).
Tuttavia non appena entriamo in multistreaming l'approssimazione perde di validità, in quanto non prevede
effetti di accelerazione gravitazionale.

Muoviamo innanzitutto dal modello cosmologico Einstein-De Sitter, che pone la curvatura dell'universo
e la costante cosmologica $\Lambda$ pari a 0 e prevede uno spazio composto sostanzialmente di CDM.
In tale cornice definiamo una mappa Lagrangiana q $\mapsto$ x(q, $\tau$) che connette la posizione
iniziale alla posizione corrente x al tempo comovente $\tau$ $\propto$ $t^{\frac{2}{3}}$.
Inoltre le coordinate x sono coordinate comoventi legate alle coordinate fisiche r dalla relazione
x = r/$a$, dove $a$ rappresenta il fattore cosmico di scala, che corrisponde a $\tau$ in un universo
EdS.
%Definiamo infine $u(x(q, \tau), \tau) = \partial_a x(q, \tau) = \partial_{\tau}(q, \tau) = \dot{x}(q, \tau)$
Definiamo infine le velocità $\bm{v} = a \dot{\bm{x}}$ e $\bm{w} = \dot{\bm{r}} = H \bm{r} + v$

Adottiamo la descizione Newtoniana di un fluido di materia oscura fredda autogravitante: in particolare
sposiamo l'ipotesi di entropia costante e di assenza di termini di pressione dal momento che trattiamo 
la CDM come una polvere non collidente. Il set di equazioni opportuno è dato da

\begin{equation}
    \label{eqn:continuità}
    \frac{\partial\rho}{\partial t}\biggr|_{\bm{r}} + \nabla_{\bm{r}}(\rho \bm{w}) = 0
\end{equation}

\begin{equation}
    \label{eqn:eulero}
    \frac{\partial\bm{w}}{\partial t}\biggr|_{\bm{r}} + (\bm{w}\cdot\nabla_{\bm{r}})\bm{w} = - \nabla_{\bm{r}} \Phi
\end{equation}

\begin{equation}
    \label{eqn:poisson}
    \laplacian_{\bm{r}}\Phi = 4\pi G \rho
\end{equation}
dove la \ref{eqn:continuità} è l'equazione di continuità che costituisce la conservazione
della massa, la \ref{eqn:eulero} è l'equazione di Eulero e viene dalla conservazione del
momento, mentre la \ref{eqn:poisson} rappresenta l'equazione di Poisson relativa
al potenziale gravitazionale $\Phi$.
Poniamo inoltre $\rho = \rho_b + \delta\rho$, dove $\rho_b$ è la densità media di background
e $\delta\rho$ costituisce una deviazione da tale valore medio. $\Phi = \Phi_b + \phi$ invece
è la somma di un potenziale di background e un potenziale peculiare $\phi$. Grazie a queste due 
apposizioni possiamo separare l'equazione di Poisson, ottenendo un equazione nella sola coordinata
comovente x.
Possiamo riscrivere anche \ref{eqn:eulero} e \ref{eqn:poisson} nelle coordinate x, usando la
relazione

\begin{equation}
    \frac{\partial}{\partial t}\biggr|_{\bm{x}} = \frac{\partial}{\partial t}\biggr|_{\bm{r}} + H(\bm{r} \cdot \nabla_{\bm{r}})
\end{equation}

Si ricava dunque

\begin{equation}
    \frac{\partial\rho}{\partial t} + 3H\rho +\frac{1}{2}\nabla_{\bm{x}}(\rho\bm{v})
\end{equation}

\begin{equation}
    \frac{\partial \bm{v}}{\partial t} + H \bm{v} + \frac{1}{a}(\bm{v}\cdot\nabla_{\bm{x}})\bm{v} = -\frac{1}{a}\nabla_{\bm{x}}\phi
\end{equation}

\begin{equation}
    \label{eqn:poissonx}
    \laplacian_{\bm{x}}\phi = 4\pi G \delta\rho
\end{equation}
A questo punto operiamo un cambio di variabili

\section{Parte seconda}

Nella presente trattazione si fa uso di coordinate Lagrangiane per impostare la descrizione di dell'instabilità
gravitazionale: in tale cornice l'approssimazione di Zel'dovich rappresenta lo sviluppo all'ordine zero delle
equazioni della fluidodinamica in regime di single-stream.
Tuttavia gli effetti gravitazionali che emergono con il multistreaming non sono descrivibili nemmeno utilizzando
ordini superiori della teoria perturbativa lagrangiana, dal momento che le equazioni del fluido cessano di essere
valide al primo shell-crossing. Pertanto la dinamica della CMD è basata unicamente sull'equazione di Vlasov, che 
sappiamo corretta ma di difficile risoluzione. Occorre quindi una teoria che risolva tale equazione, identificandone
in particolare le singolarità matematiche. Per la risoluzione assumiamo un universo unidimensionale: questa assunzione
si può giustificare sulla base del fatto che in effetti anche in tre dimensioni i fenomeni di shell-crossing hanno 
origine unidimensionale con la formazione dei pancakes (cosa sono, come si legano alle caustiche?...)

Tuttavia l'analisi offerta sarà facilmente estendibile a più dimensioni.



