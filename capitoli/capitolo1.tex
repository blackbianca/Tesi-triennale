\begin{comment}
    alcune idee/frasi da inserire:

    Le singolarità sono una caratteristica essenziale di molte discipline, e hanno talvolta conseguenze
    drammatiche sul modello matematico\dots

    Per la risoluzione assumiamo un universo unidimensionale: questa assunzione
    si può giustificare sulla base del fatto che in effetti anche in tre dimensioni i fenomeni di shell-crossing hanno 
    origine unidimensionale con la formazione dei pancakes-
    
    Tuttavia l'analisi offerta sarà facilmente estendibile a più dimensioni.
    
\end{comment}



\section{Equazioni della fluidodinamica}



Partiamo dall'ipotesi di universo primordiale omogeneo, come ci suggerisce lo spettro di CMB (FIGURA?).
Ci muoviamo innanzitutto dal modello cosmologico Einstein-De Sitter, che pone la curvatura dell'universo
e la costante cosmologica $\Lambda$ pari a 0 e prevede uno spazio composto sostanzialmente di materia
oscura fredda (CDM).
In tale cornice definiamo una mappa Lagrangiana $\mathbb{M}: \bm{q} \mapsto \bm{x}(\bm{q}, \tau)$ che connette la posizione
iniziale alla posizione corrente x al tempo di scala $\tau$ $\propto$ $t^{\frac{2}{3}}$.
Inoltre le coordinate x sono coordinate comoventi legate alle coordinate fisiche r dalla relazione
x = r/$a$, dove $a$ rappresenta il fattore cosm
ico di scala, che corrisponde a $\tau$ in un universo
EdS.
Definiamo infine le velocità $\bm{v} = a \dot{\bm{x}}$ e $\bm{w} = \dot{\bm{r}} = H \bm{r} + v$

Per approcciare la dinamica di particelle non collidenti ("polvere"), definiamo una funzione $f(\bm{x}, \bm{p}, t)$ 
che corrisponde alla densità degli stati nello spazio delle fasi. E' possibile quindi utilizzare il Teorema di 
Liouville, che afferma che la densità sopra citata si conserva nell'evoluzione di un sistema conservativo: in
effetti l'ipotesi di assenza di collisioni ci permette di soddisfare ai requisiti del teorema, e quindi 
possiamo porre a zero la derivata totale della funzione densità, ricavando l'equazione di Vlasov.

\begin{equation}
    \label{eqn:vlasov}
    \frac{\partial f}{\partial t} + \dot{\bm{x}} \nabla_{\bm{x}}f + \dot{\bm{p}} \nabla_{\bm{p}}f  = 0
\end{equation}

L'equazione di Vlasov è molto difficile da risolvere analiticamente: adottiamo quindi un approccio 
teorico semplificato, cioè la descizione Newtoniana di fluido: in particolare sposiamo l'ipotesi 
di entropia costante e di assenza di termini di pressione dal momento che trattiamo la CDM come 
una polvere autogravitante e non collidente.

Il set di equazioni adatto all'approccio fluidodinamico è dato da

\begin{equation}
    \label{eqn:continuità}
    \frac{\partial\rho}{\partial t}\biggr|_{\bm{r}} + \nabla_{\bm{r}}(\rho \bm{w}) = 0
\end{equation}

\begin{equation}
    \label{eqn:eulero}
    \frac{\partial\bm{w}}{\partial t}\biggr|_{\bm{r}} + (\bm{w}\cdot\nabla_{\bm{r}})\bm{w} = - \nabla_{\bm{r}} \Phi
\end{equation}

\begin{equation}
    \label{eqn:poisson}
    \laplacian_{\bm{r}}\Phi = 4\pi G \rho
\end{equation}
dove la \ref{eqn:continuità} è l'equazione di continuità che costituisce la conservazione
della massa, la \ref{eqn:eulero} è l'equazione di Eulero e viene dalla conservazione del
momento, mentre la \ref{eqn:poisson} rappresenta l'equazione di Poisson relativa
al potenziale gravitazionale $\Phi$.

Poniamo inoltre $\rho = \rho_b + \delta\rho$, dove $\rho_b$ è la densità media di background
e $\delta\rho$ costituisce una deviazione da tale valore medio. $\Phi = \Phi_b + \phi{'}$ invece
è la somma di un potenziale di background e un potenziale peculiare $\phi{'}$. Grazie a queste due 
apposizioni possiamo separare l'equazione di Poisson, ottenendo un equazione nella sola coordinata
comovente x.
Possiamo riscrivere anche \ref{eqn:eulero} e \ref{eqn:poisson} nelle coordinate x, usando la
relazione

\begin{equation}
    \frac{\partial}{\partial t}\biggr|_{\bm{x}} = \frac{\partial}{\partial t}\biggr|_{\bm{r}} + H(\bm{r} \cdot \nabla_{\bm{r}})
\end{equation}

Si ricava dunque

\begin{equation}
    \label{eqn:continuitax}
    \frac{\partial\rho}{\partial t} + 3H\rho +\frac{1}{2}\nabla_{\bm{x}}(\rho\bm{v}) = 0
\end{equation}

\begin{equation}
    \label{eqn:eulerox}
    \frac{\partial \bm{v}}{\partial t} + H \bm{v} + \frac{1}{a}(\bm{v}\cdot\nabla_{\bm{x}})\bm{v} = -\frac{1}{a}\nabla_{\bm{x}}\phi = 0
\end{equation}

\begin{equation}
    \label{eqn:poissonx}
    \laplacian_{\bm{x}}\phi{'} = 4\pi G \delta\rho
\end{equation}

Le equazioni della fluidodinamica rappresentano in effetti uno sviluppo dell'equazione 
di Vlasov fino al primo ordine. Per spiegare questo passaggio, osserviamo che la densità
di massa e la velocità sono associate rispettivamente al momento di aspettazione di ordine 
zero e di primo ordine della densità nello spazio delle fasi $f(\bm{x}, \bm{p}, t)$.

\begin{equation}
    \label{eqn:rho}
    \rho(\bm{x}, t) = \frac{m}{a^3}\int d^3p f(\bm{x}, \bm{p}, t)
\end{equation}

\begin{equation}
    \label{eqn:vel}
    \bm{v}(\bm{x}, t) = \frac{m}{a^3}\frac{\int d^3p \bm{p}f(\bm{x}, \bm{p}, t)}{\int d^3p f(\bm{x}, \bm{p}, t)}
\end{equation}
Integrando ora l'equazione di Vlasov sul dominio del momento $\bm{p}$, si trova che l'ultimo termine
dell'integrando rappresenta un'integrale di volume della forza $\partial f$/$\partial p$, che tramite
il Teorema di Gauss si può riscrivere come un integrale su una superficie all'infinito, dove la forza
si annulla. Utilizzando poi le definizioni \ref{eqn:rho} e \ref{eqn:vel} si ricava 
\begin{equation}
    \frac{\partial}{\partial t}(a^3 \rho) + \frac{1}{a^2}\nabla_x \int d^3 p \bm{p} f = 0
\end{equation}
maneggiando opportunamente quest'ultima e utilizzando le definzioni fornite in \ref{eqn:rho} e \ref{eqn:vel}
si arriva esattamente all'equazione di continuità \ref{eqn:continuità}.

Se invece si moltiplica l'equazione di Vlasov per $\bm{p}$ per poi integrare di nuovo su tale variabile, 
conviene lavorare sul termine i-esimo e operare un'integrazione per parti sempre sull'ultimo addendo dell'
integrando.

\begin{equation}
    \frac{\partial }{\partial t} \int d^3 p p^i f + \frac{1}{ma^2} \partial^i \int d^3 p p_j p_j f + a^3 \rho \partial^i \phi = 0 
\end{equation}

Manipolando questa espressione e utilizzando l'equazione di continuità, si arriva proprio all'equazione di Eulero
\ref{eqn:eulero}.

Le equazioni della fluidodinamica rappresentano dunque i primi termini dello sviluppo dell'equazione di Vlasov,
e perciò costituiscono una via più facilmente percorribile, offrendo la possibilità di giungere a delle soluzioni
analitiche altrimenti proibitive.


\section{Approssimazione di Zel'dovich}

A questo punto tuttavia conviene operare un ulteriore cambio di variabili sul set di equazioni ottenute \ref{eqn:continuitax},
\ref{eqn:eulerox} e \ref{eqn:poissonx}. Per farlo si definisce meglio il fattore di scala per mezzo di un'ampiezza $a_{*}$ e
un tempo caratteristico $t_{*}$, in modo che $a(t) = (t / t_{*})^{2/3}$. Ricordando inoltre $\rho = \rho_b + \delta\rho$ e 
$\bm{v} = a \dot{\bm{x}}$, facciamo le seguenti sostituzioni

\begin{gather}
    \rho \mapsto \eta = \frac{\rho}{\rho_b} = 1 + \delta \\
    \bm{v} \mapsto \bm{u} = \frac{d\bm{x}}{da} = \frac{d\bm{x}}{dt}\frac{dt}{da} = \frac{\bm{v}}{a\dot{a}} \\
    \phi{'} \mapsto \phi = \frac{3t_{*}^2}{2a_{*}^3}\phi{'}
\end{gather}
Grazie alla mappatura $(\rho, \bm{v}, \phi{'})\mapsto(\eta, \bm{u}, \phi)$, le equazioni del fluido assumono
la nuova forma 

\begin{gather}
    \frac{D\bm{u}}{Da} + \frac{3}{2a}\bm{u} = -\frac{3}{2a}\nabla\phi \\
    \frac{D\eta}{Da} + \eta \nabla \cdot \bm{u} = 0 \\
    \laplacian \phi = \frac{\delta}{a}
\end{gather}
dove la derivata $D/Da$ si dice \textit{derivata convettiva}.
Ora usiamo il fatto che in un universo EdS linearizzato, la soluzione \textit{growing mode} è data complessivamente
da  $\delta \propto t^{\frac{2}{3}}$, $\bm{v} \propto t^{\frac{1}{3}}$ e $\phi = const$. Con questi andamenti, è
evidente che la nuova coordinata di velocità $\bm{u} = \bm{v} / (a\dot{a}) \approx const$, dal momento che sia 
$\bm{v}$ che $a\dot{a} \propto t^{2/3} * t^{-1/3}$ hanno lo stesso andamento $t^{1/3}$.

Ma allora il termine $Du/Da$ nell'equazione di Eulero si può considerare nullo
\begin{equation}
    \label{eqn:zel}
    \frac{Du}{Da} = 0
\end{equation}
ottenendo così

\begin{equation}
    \label{eqn:zelsol}
    \bm{u} = -\nabla\phi
\end{equation}
che è la soluzione linearizzata, valida per piccole deviazioni dalla densità di background, ovvero per $\delta < 1$.
L'approssimazione di Zel'dovich sta nel considerare tale risultato legittimo anche oltre il regime di linearità, ossia 
assumere la validità di \ref{eqn:zel} ovunque.
Si può inoltre osservare che in queste condizioni l'equazione di Poisson gravitazionale risulta disaccoppiata dalle
altre due ed è utile per applicare le condizioni iniziali.
La \ref{eqn:zel} descrive un moto rettilineo uniforme, in cui la particella è soggetta solamente alla propria inerzia
senza perturbazioni gravitazionali esterne. Quindi se la posizione iniziale è descritta dalla coordinata lagrangiana
$\bm{q}$, allora per ogni posizione euleriana $\bm{x}$ del moto varrà che 

\begin{equation}
    \bm{u}(\bm{x}, a) = \bm{u}_0(\bm{q})
\end{equation}
Le traiettorie delle particelle, rettilinee e a velocità costante $u$, sono descritte da

\begin{equation}
    \bm{x}(\bm{q, a}) = \bm{q} + a u_0(\bm{q})
\end{equation}
Ma usando \ref{eqn:zelsol} si ricaverà

\begin{equation}
    \bm{x}(\bm{q, a}) = \bm{q} - a \nabla\phi_0
\end{equation}

Alla fine di questi passaggi si è in grado di identificare la mappa $\mathbb{M}$ che collega le coordinate lagrangiane 
iniziali con quelle finali euleriane. 

A questo punto saremmo in grado di risolvere l'equazione di continuità semplicemente come un'equazione a variabili
separabili, trovando quindi la forma di $\eta$. Tuttavia la via più semplice risulta invece dall'impostare la 
conservazione della massa dei singoli elementi fluidi.

\begin{equation}
    \eta(\bm{x}, a)d^3x=\eta_0(\bm{q})d^3q
\end{equation}
da cui

\begin{equation}
    \eta(\bm{x}(\bm{q}, a), a) = (1 + \delta_0(\bm{q}))\det\left(\frac{\partial q}{\partial x}\right) 
\end{equation}
Ma supponendo che nella configurazione iniizale la perturbazione di energia sia nulla $\delta_0 = 0$
e contemporaneamente utilizzando le proprietà del determinante, si potrà scrivere anche

\begin{equation}
    \eta(\bm{x}(\bm{q}, a), a) = \left[\det\left(\frac{\partial x}{\partial q}\right)\right]^{-1}
\end{equation}
E' possibile scrivere la matrice $\partial x$/$\partial q$ in componenti, sapendo che $x_i = q_i - a \frac{\partial\phi_0}{\partial q^i}$, 
e derivando ulteriormente 

\begin{equation}
    \frac{\partial x^i}{\partial q^j} = \delta^i_j - a \frac{\partial^2 \phi_0}{\partial q_i \partial q^j} = 
    \delta^i_j - a D^i_j(q)
\end{equation}
dove si è definito il tensore di deformazione $D^i_j$. Nel sistema di riferimento opportuno tale tensore ha forma
diagonale, con i tre autovalori $\lambda_1(\bm{q})$, $\lambda_2(\bm{q})$ e $\lambda_3(\bm{q})$, dipendenti dalle
coordinate iniziali. Si può dimostrare che nell'ipotesi in cui il potenziale $phi_0$ sia gaussiano come previsto
dal meccanismo di inflazione, allora nel almeno uno dei tre autovalori del tensore di deformazione è positivo 
nel 92$\%$ dei casi.
Ora, riscrivendo il rapporto di densità $\eta$

\begin{multline}
    \eta(\bm{x}(\bm{q}, a), a) = \left[\det\left(\frac{\partial x}{\partial q}\right)\right]^{-1} = \left[\det\left( \mathbb{I}- a D \right)\right]^{-1} = \\
    = \frac{1}{(1-a\lambda_1(\bm{q}))(1-a\lambda_2(\bm{q}))(1-a\lambda_3(\bm{q}))}
\end{multline}

A questo punto, è evidente che, supponendo che $\lambda_1(\bm{q})$ sia l'autovalore maggiore, allora il tempo 
$\bar{a}=1/\lambda_1(\bm{q})$ rappresenta una criticità, in quanto la quantità $\eta$ diverge e con essa la 
densità. Questo tipo di evento è detto \textit{shell-crossing} o \textit{caustica} ed è rappresenta il 
fenomeno che si registra quando due particelle con diverse coordinate lagrangiane iniziali 
$\bm{q}_1$ e $\bm{q}_2$ confluiscono nella stessa coordinata di campo $\bm{x}$, dando luogo a densità
infinita. 
Tale divergenza è racchiusa nella mancata biunivocità della mappa $\mathbb{M}^{-1}$, dal momento che a 
una sola coordinata euleriana possono corrispondere più coordinate iniziali. A causa della mancanza
di tale biunivocità, la matrice Jacobiana $\partial x/\partial q$ risulta essere mal definita.
Le caustiche delimitano le zone del cosiddetto \textit{multistreaming}, regioni entro le quali non è
più possibile ritracciare le particelle all'indietro, in quanto non sappiamo come si sono comportate
negli eventi di shell-crossing. Le strutture di materia che si formano sono detti \textit{pancakes},
filamenti "quasi" unidimensionali o bidimenzionali, nel senso che la grumosità si sviluppa 
sostanzialmente in una oppure due direzioni. Queste strutture saranno l'opportuna sede la formazione
delle galassie.

Nel regime di single-stream l'approssimazione di Zel'dovich è opportuna, in quanto costitusice 
un modello non accelerato dove le particelle procedono imperturbate lungo una traiettoria rettilinea
 e a velocità costante. Tuttavia a partire dal primo shell-crossing l'approssimazione perde di validità, 
in quanto non prevede effetti di accelerazione gravitazionale esercitata dalle particelle vicine, che 
modificherebbe il cammino della particella in modo sensibile. 

\section{Metodi di ricostruzione}














\begin{comment}
SCALETTA SEGUENTE
nuova sezione con carrellata su metodi di ricostruzione (citare Mohayaee)
- cos'è la reconstruction
- Peebles
- POTENT
- MAK, particolare attenzione perché prende le origin dalla trattazione fluido
ed è il metodo che fornisce unicità
\end{comment}