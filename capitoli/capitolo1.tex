\section{Parte prima}

Le singolarità sono una caratteristica essenziale di molte discipline, e hanno talvolta conseguenze
drammatiche sul modello matematico\dots

In cosmologia, le caustiche sono i processi fondanti della formazione delle strutture cosmologiche
su larga scala. Le caustiche corrispondono ai punti presso i quali la densità risulta localmente infinita
e sono i prodotti del cosiddetto \textit{shell-crossing}, che si verifica all'incrociarsi delle traiettorie
delle particelle di CDM. (definire meglio caustiche...).
In particolare con il primo incontro delle lo traiettorie le particelle si muovo dal regime di \textit{single-stream}
a quello di \textit{multistreaming}, dove le orbite particellari si intersecano e i fenomeni di shell-crossing
si moltiplicano, determinando il diramarsi dei flussi.

Nel regime di single-stream l'approssimazione di Zel'dovich è opportuna, in quanto costitusice un modello non
accelerato dove le particelle procedono imperturbate e a velocità costante (quale velocità, comovente?).
Tuttavia non appena entriamo in multistreaming l'approssimazione perde di validità, in quanto non prevede
effetti di accelerazione gravitazionale.

Nella seguente trattazione si fa uso di coordinate Lagrangiane per impostare la descrizione di dell'instabilità
gravitazionale: in tale cornice l'approssimazione di Zel'dovich rappresenta lo sviluppo di ordine zero delle
equazioni della fluidodinamica, in particolare dell'equazione di Vlasov, in regime di single-stream.
Tuttavia gli effetti gravitazionali che emergono con il multistreaming non sono descrivibili nemmeno utilizzando
ordini superiori della teoria perturbativa lagrangiana, dal momento che le equazioni del fluido cessano di essere
valide al primo shell-crossing. Pertanto la dinamica della CMD è basata unicamente sull'equazione di Vlasov, che 
sappiamo corretta ma di difficile risoluzione. Occorre quindi una teoria che risolva tale equazione, identificandone
in particolare le singolarità matematiche. Per la risoluzione assumiamo un universo unidimensionale: questa assunzione
si può giustificare sulla base del fatto che in effetti anche in tre dimensioni i fenomeni di shell-crossing hanno 
origine unidimensionale con la formazione dei pancakes (cosa sono, come si legano alle caustiche?...)

Tuttavia l'analisi offerta sarà facilmente estendibile a più dimensioni.

\section{Parte seconda}



