\begin{comment}
    alcune idee/frasi da inserire


\end{comment}



\section{Setup e scelta delle condizioni iniziali}

In cosmologia le caustiche, che nascono da una singolarità nella trattazione di fluido esposta nel precedente
capitolo, rappresentano in realtà il processo fondamentale della formazione di strutture a grande scala 
nell'universo. Al primo shell-crossing le particelle affrontano il cosiddetto \textit{collaso gravitazionale
secondario}, per cui il numero di collisioni aumenta sostanzialmente, e così aumenta il numero di pancakes e 
strutture filiformi. Tuttavia la forma oblata quasi unidimensionale del pancake non è l'unica forma che si 
registra: infatti nel contesto di una trattazione nonlineare della dinamica gravitazionale con l'approssimazione
di Zel'dovich si incontrano una serie di strutture scrupolosamente classificate per i casi semplici unidimensionale
e bidimensionale in \cite{arnold}.

Tuttavia le strutture che possono emerge a seguito del collasso gravitazionale non possono essere previste 
da un modello che include l'approssimazione di Zel'dovich, che si basa sull'ipotesi di velocità costante.
In effetti si è spiegato nel primo capitolo che ZA risulta essere una soluzione esatta solamente nel caso 
unidimensionale, e in ogni caso valida esclusivamente fino al primo shell-crossing, dove le equazioni del
fluido falliscono. \'E necessario quindi tornare all'equazione di riferimento, ossia all'equazione di Vlasov 
\ref{eqn:vlasov}, anche limitandosi in prima battuta al caso unidimensionale, dal momento che gli 
shell-crossing si manifestano, almeno nella loro fase iniziale, come fenomeni unidimensionali.

Inquadriamo l'analisi in un universo EdS, dove $\tau = a$ come accennato in precedenza, quando abbiamo anche
fornito la definizione $u(x(q, \tau), \tau) = \partial_a x(q, \tau) = \partial_{\tau}(q, \tau) = \dot{x}(q, \tau)$.
Si riportano le equazioni \ref{eqn:cont_tau} e \ref{eqn:poisson_tau}.

\begin{gather}
    \label{eqn:cont_new}
    \ddot{x} + \frac{3}{2\tau}\dot{x} = - \frac{3}{2\tau}\nabla_x \phi \\
    \label{eqn:poisson_new}
    \laplacian_x\phi = \frac{\delta}{\tau}
\end{gather}
dove si ricorda che $\delta$ rappresenta la deviazione dalla densità media, e sia $\delta$
che $\phi$ dipendono da x.
Si osserva ora che tale set di equazioni gode di invarianza rispetto a trasformazioni di 
Galileo della forma $x \mapsto x + \zeta(\tau)$: possiamo sfruttare al presenza di questa 
simmetria per imporre una condizione al centro di massa. Per scriverla definiamo il dislocamento
lagrangiano $\xi(q, \tau):= x(q, \tau) - q$.

\begin{equation}
    \int_{\mathbb{T}}\xi(q', \tau) dq' = 0
\end{equation}
Se tale condizione non fosse rispettata significherebbe che esiste un certa direzione
privilegiata di moto delle particelle, incompatibilmente con l'assenza di forze esterne.

A questo punto si procede prendendo la divergenza di \ref{eqn:cont_new} e inserendovi 
\ref{eqn:poisson_new}, ottenendo

\begin{equation}
    \nabla \ddot{x} + \nabla \left(\frac{3}{2\tau}\dot{x}\right) = -\frac{3}{2\tau}\laplacian\phi = -\frac{3\delta(x)}{2\tau^2}
\end{equation}

Sviluppando i calcoli si trova che questa equazione si può riscrivere come 

\begin{equation}
    \partial_q \mathcal{R}_{\tau}\xi = -\frac{3}{2}F(x(q, \tau))
\end{equation}














\section{Parte seconda}

